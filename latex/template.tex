\documentclass{article}

\usepackage{amsmath}
\usepackage{amssymb}
\usepackage{hyperref}
\usepackage{tabu}
\usepackage{graphicx}
\usepackage{placeins}
\usepackage[margin=1in]{geometry}
\usepackage{float}
\usepackage{mathtools}
\usepackage{textcomp}
\usepackage{pdfpages}
\usepackage{minted}
\usepackage[style=iso]{datetime2}
\usepackage{bbm}
\usepackage[flushleft]{threeparttable}
\usepackage{tikz}
\usetikzlibrary{bayesnet}
\usepackage{braket}
\usepackage{cancel}
\usepackage{enumitem}
\usepackage{subcaption}
\usepackage{wrapfig}
\usepackage{titling}
\usepackage[font=small,labelfont=bf]{caption}
\usepackage{subcaption}
\usepackage{syntax}
\usepackage{listings}
\definecolor{mygreen}{rgb}{0,0.6,0}
\definecolor{myorange}{rgb}{1.0,0.5,0.3}
\definecolor{mymauve}{rgb}{0.58,0,0.82}
\definecolor{myblue}{rgb}{0.05,0.19,0.57}
\definecolor{mygrey}{rgb}{0.4,0.4,0.4}
\definecolor{myred}{rgb}{0.9,0.2,0.15}

\lstdefinelanguage{pddl}
{
  sensitive=false,    % not case-sensitive
  morecomment=[l]{;}, % line comment
  alsoletter={:,-},   % consider extra characters
  morekeywords={
    define,domain,problem,not,and,or,when,forall,exists,either,
    :domain,:extends,:requirements,:types,:objects,:constants,
    :predicates,:action,:parameters,:precondition,:effect,:functions,
    :fluents,:primary-effect,:side-effect,:init,:goal,assign
    :strips,:adl,:equality,:typing,:conditional-effects, :metric, minimize,
    :negative-preconditions,:disjunctive-preconditions,
    :existential-preconditions,:universal-preconditions
  },
  keywords=[2]{object}, % Objects
  keywords=[3]{}, % Predicates
  keywords=[4]{assignMenus, assignMC, assignSC}, % Actions
  keywords=[5]{} % Functions
}

\lstset
{
  language={pddl},
  basicstyle=\small\ttfamily, % Global Code Style
  captionpos=b, % Position of the Caption (t for top, b for bottom)
  extendedchars=true, % Allows 256 instead of 128 ASCII characters
  tabsize=2, % number of spaces indented when discovering a tab 
  columns=fixed,
  keepspaces=true,
  showstringspaces=false,
  breaklines=true,
  numberstyle=\tiny\ttfamily, % style of the line numbers
  commentstyle=\color{mygrey}, % style of comments
  keywordstyle=\color{mygreen}, % style of keywords
  keywordstyle=[2]\color{mymauve},
  keywordstyle=[3]\color{myorange},
  keywordstyle=[4]\color{myblue},
  keywordstyle=[5]\color{myred},
  stringstyle=\color{blue}, % style of strings
}

\lstset{
    language=PDDL,
    escapeinside={(*}{*)},
}
\DeclareFontShape{OT1}{cmtt}{bx}{n}{<5><6><7><8><9><10><10.95><12><14.4><17.28><20.74><24.88>cmttb10}{}


\usepackage[american]{babel}
\usepackage{csquotes}
% \usepackage[style=apa,backend=biber]{biblatex}
% \DeclareLanguageMapping{american}{american-apa}
% \bibliography{references} % .bib file
% \nocite{*}

\setlength{\droptitle}{-8em}

% \bibliographystyle{apacite}

\DeclareMathOperator*{\argmin}{arg\,min}
\DeclareMathOperator*{\argmax}{arg\,max}
\DeclareMathOperator*{\EV}{E}
\DeclareMathOperator*{\var}{var}
\DeclareMathOperator*{\tr}{tr}
\DeclareMathOperator*{\mathspan}{span}
\newcommand{\MAT}[1]{\begin{bmatrix} #1 \end{bmatrix}}
\newcommand{\sMAT}[1]{\left(\begin{smallmatrix} #1 \end{smallmatrix}\right)}
\DeclareMathOperator*{\vecop}{vec}
\newcommand{\norm}[1]{\left\Vert #1 \right\Vert}
\newcommand{\fnorm}[1]{\left\Vert #1 \right\Vert_F}
\newcommand{\indep}{\perp \!\!\! \perp}
\newcommand{\sol}{{\bf Solution: }}
\newcommand{\soln}{\sol}
\newcommand{\solution}{\sol}
\newcommand{\TODO}{{\bf \color{red} TODO: THIS}}

% \renewcommand{\dateseparator}{--}

\newcommand{\figref}[1]{Figure~\ref{#1}}

\pagestyle{empty} \addtolength{\textwidth}{1.0in}
\addtolength{\textheight}{0.5in} \addtolength{\oddsidemargin}{-0.5in}
\addtolength{\evensidemargin}{-0.5in}
\newcommand{\ruleskip}{\bigskip\hrule\bigskip}
\newcommand{\nodify}[1]{{\sc #1}} \newcommand{\points}[1]{{\textbf{[#1
points]}}}

\newcommand{\bitem}{\begin{list}{$\bullet$}%
{\setlength{\itemsep}{0pt}\setlength{\topsep}{0pt}%
\setlength{\rightmargin}{0pt}}} \newcommand{\eitem}{\end{list}}

\newcommand{\G}{\mathcal{G}}
\newcommand{\E}{\mathbb{E}}
\newcommand{\R}{\mathbb{R}}
\newcommand{\LL}{\mathcal{L}}

%\newcommand{\bE}{\mbox{\boldmath $E$}}
%\newcommand{\be}{\mbox{\boldmath $e$}}
%\newcommand{\bU}{\mbox{\boldmath $U$}}
%\newcommand{\bu}{\mbox{\boldmath $u$}}
%\newcommand{\bQ}{\mbox{\boldmath $Q$}}
%\newcommand{\bq}{\mbox{\boldmath $q$}}
%\newcommand{\bX}{\mbox{\boldmath $X$}}
%\newcommand{\bY}{\mbox{\boldmath $Y$}}
%\newcommand{\bZ}{\mbox{\boldmath $Z$}}
%\newcommand{\bx}{\mbox{\boldmath $x$}}
%\newcommand{\by}{\mbox{\boldmath $y$}}
%\newcommand{\bz}{\mbox{\boldmath $z$}}

\newcommand{\true}{\mbox{true}}
\newcommand{\Parents}{\mbox{Parents}}

\newcommand{\ww}{{\bf w}}
\newcommand{\xx}{{\bf x}}
\newcommand{\yy}{{\bf y}}
\newcommand{\real}{\ensuremath{\mathbb{R}}}


\newcommand{\eat}[1]{}

\newcommand{\CInd}[3]{({#1} \perp {#2} \mid {#3})}
\newcommand{\Ind}[2]{({#1} \perp {#2})}

\setlength{\parindent}{0pt} \setlength{\parskip}{0.5ex}

% \title{Homework 12} %  \vspace{-3cm} % before name to raise
% \subtitle{MATH-GA 2110}
% \vspace{-5cm} % before name to raise
\title{Game Creation DSL}
\author{Guy Davidson}
% \date{\today}


\begin{document}
\maketitle

{{BODY}}

\section{Modal Definitions in Linear Temporal Logic}
\subsection{Linear Temporal Logic definitions}
Linear Temporal Logic (LTL) offers the following operators, and using $\varphi$ and $\psi$ as the symbols (in our case, predicates). 
I'm trying to translate from standard logic notation to something that makes sense in our case, where we're operating sequence of states $S_0, S_1, \cdots, S_n$. 
\begin{itemize}
    \item \textbf{Next}, $X \psi$: at the next timestep, $\psi$ will be true. If we are at timestep $i$, then $S_{i+1} \vdash \psi$
    
    \item \textbf{Finally}, $F \psi$: at some future timestep, $\psi$ will be true. If we are at timestep $i$, then $\exists j > i:  S_{j} \vdash \psi$
    
    \item \textbf{Globally}, $G \psi$: from this timestep on, $\psi$ will be true. If we are at timestep $i$, then $\forall j: j \geq i: S_{j} \vdash \psi$
    
    \item \textbf{Until}, $\psi U \varphi$: $\psi$ will be true from the current timestep until a timestep at which $\varphi$ is true. If we are at timestep $i$, then $\exists j > i: \forall k: i \leq k < j: S_k \vdash \psi$, and $S_j \vdash \varphi$.
    \item \textbf{Strong release}, $\psi M \varphi$: the same as until, but demanding that both $\psi$ and $\varphi$ are true simultaneously: If we are at timestep $i$, then $\exists j > i: \forall k: i \leq k \leq j: S_k \vdash \psi$, and $S_j \vdash \varphi$. 
    
    \textit{Aside:} there's also a \textbf{weak until}, $\psi W \varphi$, which allows for the case where the second is never true, in which case the first must hold for the rest of the sequence. Formally, if we are at timestep $i$, \textit{if} $\exists j > i: \forall k: i \leq k < j: S_k \vdash \psi$, and $S_j \vdash \varphi$, and otherwise, $\forall k \geq i: S_k \vdash \psi$. Similarly there's \textbf{release}, which is the similar variant of strong release. I'm leaving those two as an aside since I don't know we'll need them. 
    
\end{itemize}

\subsection{Satisfying a (then ...) operator}
Formally, to satisfy a preference using a (then ...) operator, we're looking to find a sub-sequence of $S_0, S_1, \cdots, S_n$ that satisfies the formula we translate to. 
We translate a (then ...) operator by translating the constituent sequence-functions (once, hold, while-hold)\footnote{These are the ones we've used so far in the interactive experiment dataset, even if we previously defined other ones, too.} to LTL. 
Since the translation of each individual sequence function leaves the last operand empty, we append a `true' ($\top$) as the final operand, since we don't care what happens in the state after the sequence is complete. 

(once $\psi$) := $\psi X \cdots$

(hold $\psi$) := $\psi U \cdots$

(hold-while $\psi$ $\alpha$ $\beta$ $\cdots \nu$) := ($\psi M \alpha) X (\psi M \beta) X \cdots X (\psi M \nu) X \psi U \cdots$ where the last $\psi U \cdots$ allows for additional states satisfying $\psi$ until the next modal is satisfied.

For example, a sequence such as the following, which signifies a throw attempt:
\begin{lstlisting}
(then
    (once (agent_holds ?b))
    (hold (and (not (agent_holds ?b)) (in_motion ?b))) 
    (once (not (in_motion ?b)))
)
\end{lstlisting}
Can be translated to LTL using $\psi:=$ (agent_holds ?b), $\varphi:=$ (in_motion ?b) as:

$\psi X (\neg \psi \wedge \varphi) U (\neg \varphi) X \top $

Here's another example: 
\begin{lstlisting}
(then 
    (once (agent_holds ?b))  (* \color{blue} $\alpha$*)
    (hold-while 
        (and (not (agent_holds ?b)) (in_motion ?b)) (* \color{blue} $\beta$ *)
        (touch ?b ?r) (* \color{blue} $\gamma$*)
    ) 
    (once  (and (in ?h ?b) (not (in_motion ?b)))) (* \color{blue} $\delta$*) 
)
\end{lstlisting}
If we translate each predicate to the letter appearing in blue at the end of the line, this translates to:

$\alpha X (\beta M \gamma) X \beta U \delta X \top$

\subsection{Satisfying (at-end ...) or (always ...) operators}
Thankfully, the two other types of temporal specifications we find ourselves using as part of preferences are simpler to translate. 
Satisfying an (at-end ...) operator does not require any temporal logic, since the predicate it operates over is evauated at the terminal state of gameplay.
The (always ...) operator is equivalent to the LTL globally operator: (always $\psi$) := $G \psi$, with the added constraint that we begin at the first timestep of gameplay. 


% I'll attempt to check slightly more formally at some point, but I don't think we end up with many structures that are more complex than this. 
% The predicate end up being rather more complex, but that doesn't matter to the LTL translation.

% \section{Modal Definitions}

% \begin{itemize}
%     \item These definitions attempt to offer precision on how the (then ...) operator works. It receives a series of sequence-functions (once, hold, etc.), each of which is parameterized by one or more predicate conditions. 

%     \item For the inner sequence-functions, I used the parentheses notation to mean "evaluated at these timesteps" -- does this notation make sense? Should I also use it for the entire then-expression? 
    
%     \item I've only provided here the for the ones currently used in the interactive experiment. 
% \end{itemize}

% $(\text{then}\ \langle SF_1 \rangle \ \langle SF_2 \rangle \cdots \langle SF_n \rangle) := \exists t_0 \leq t_1 < t_2 < \cdots < t_n$ such that $SF_1(t_0, t_1) \land SF_2(t_1, t_2) \land \cdots \land SF_n(t_{n-1}, t_n) = \text{true}$, that is, each seq-func evaluated at these timesteps evaluates to true. 

% $(\text{once}\ \langle C \rangle)(t_{i-1}, t_i) := t_i = t_{i-1} + 1, S[t_i] \vdash C$, that is, the condition C holds at the next timestep from the previous assigned timestep.

% $(\text{hold}\ \langle C \rangle)(t_{i-1}, t_i) := \forall t:  t_{i-1} < t \leq t_i, S[t] \vdash C$, that is, the condition holds for all timesteps starting immediately after the previous timestep and until the current timestep. 

% $(\text{hold-while}\ \langle C \rangle \ \langle C_a \rangle \cdots \langle C_m \rangle)(t_{i-1}, t_i) := \forall t:  t_{i-1} < t \leq t_i, S[t] \vdash C$ and $\exists t_a, \cdots, t_m: t_{i-1} < t_a < \cdots < t_m < t_i$ such that $S[t_a] \vdash C_a, \cdots, S[t_m] \vdash C_m$, that is, the same as hold happens, and while this condition $C$ holds, there exist non-overlapping states in sequence where each of the additional conditions provided hold for at least a single state.

% $(\text{hold-for}\ \langle n \rangle \ \langle C \rangle)(t_{i-1}, t_i) := t_i \geq t_{i-1} + n, \forall t:  t_{i-1} < t \leq t_i, S[t] \vdash C$, that is, the same as the standard hold but for at least $n$ timesteps. 

% $(\text{forall-sequence}\ \langle \text{forall-quantifier(s)} \rangle \ \langle \text{then-expr} \rangle)(t_{i-1}, t_i): \forall o \in \{a, b, \cdots, k\}$ satisfying the object assignments in the forall quantifier, $\exists t_0^o, t_1^o, \cdots, t_m^o$ that satisfy the inner then expression, such that $t_{i-1} < t_0^a < \cdots t_m^a < t_0^b < \cdots t_m^b < \cdots < t_0^k < \cdots t_m^k < t_i$, that is, the series of timesteps satisfying the inner then-expression for each object assignment do not overlap, happen in sequence, and fall between the previous assigned timestep and the current assigned timestep. 

% \section{Open Questions}
% \begin{itemize}
%     \item Do we want to define syntax to quantify streaks? Some participants will use language like ``every three successful scores in a row get you a point''. An alternative to defining syntax or sequences would be to define the preference to count three successful attempts in a row, but that might be more awkward?
    
%     \item How do we want to work with type hierarchy, such as block or ball being the super-types for all blocks or balls -- is it an implicit (either ...) over all of the sub-types? Or do we want to provide the hierarchy in some way to the model, perhaps as part of the enumeration of all valid types in a given environment/scene?
    
%     \item (I'm sure there are more open questions -- will add later)
% \end{itemize}

\end{document}